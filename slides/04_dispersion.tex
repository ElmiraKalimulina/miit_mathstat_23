% Le titre de la partie
\section{Дисперсия случайной величины}

%%%%%%%%%%%%%%%%%%%%%%%%%%%%%%%%%%%%%%%%%%%%%%%%
% Première diapo avec un exemple de tableau
%%%%%%%%%%%%%%%%%%%%%%%%%%%%%%%%%%%%%%%%%%%%%%%%
\begin{frame}
\frametitle{Дисперсия случайной величины}
\framesubtitle{Определение}
\theoremstyle{definition}
\begin{definition}{Дисперсией}
 величины $\xi$ (обозначается $D \xi$) называется величина
    $D\xi = E (\xi - E \xi)^2$
\end{definition}
\pause
\begin{definition}{}
Величина $\sigma = + \sqrt{ D\xi}$ называется стандартным отклонением значений
случайной величины $\xi$ от ее среднего значения $E\xi$.
\end{definition}
\end{frame}

\begin{frame}
\frametitle{Дисперсия случайной величины}
\framesubtitle{Свойства}
Поскольку
\begin{equation*}
    E(\xi - E\xi)^2 = E(\xi^2 - 2 \xi E\xi + (E \xi)^2 ) = \pause
    E\xi^2 - (E\xi) ^2 ,
\end{equation*}
\pause
то
\begin{equation*}
    \boxed{D \xi= E\xi^2 - (E\xi) ^2}.
\end{equation*}
\pause
\begin{equation*}
    \boxed{D \xi \geq 0}~~~\text{(из определения).}
\end{equation*}
\pause
\begin{equation*}
    \boxed{D (a+b\xi) =b^2 D\xi  }~~~\text{(из определения), где $a$, $b$ -  постоянные.}
\end{equation*}
\end{frame}

\begin{frame}
\frametitle{Ковариация случайных величин}
\framesubtitle{Определение}
Пусть $\xi$ и $\eta$ - две случайные величины. Тогда
\pause
\begin{eqnarray*}
    D(\xi + \eta) = E( \xi + \eta - E (\xi +  \eta ))^2 = 
    E( ( \xi - E \xi) + (\eta -  E \eta ) )^2 = \\
    = D  \xi + D \eta + 2
    \underbrace{
    E( \xi - E \xi)( \eta - E \eta)}_{ \pause cov (\xi, \eta) }.
\end{eqnarray*}
Величина $cov (\xi, \eta)$ называется ковариацией случайных величин $\xi$ и $\eta$.
\end{frame}

\begin{frame}
\frametitle{Корреляция случайных величин}
\framesubtitle{Определение, свойства.}
Если $\xi > 0$ и $\eta > 0 $, то величина
\pause
\begin{eqnarray*}
    \rho(\xi, \eta) = \frac{cov (\xi, \eta)}{\sqrt{\xi} \sqrt{\eta}}
\end{eqnarray*}
называется коэффициентом корреляции случайных величин $\xi$ и $\eta$.
\\
\\
\pause
Если $\rho(\xi, \eta) = \pm 1$, то 
$\eta = a \xi + b$, $(\xi, \eta)$ (линейно зависимы),
где $a > 0$, если $ \rho(\xi, \eta) = +1$, и $a < 0$, если $\rho(\xi, \eta) = -1$.
\end{frame}

\begin{frame}
\frametitle{Дисперсия}
\framesubtitle{Определение, свойства, ...}
Если $\xi$ и $\eta$ независимы, то независимы $\xi-E\xi$ и $\eta-E\eta$,  и по свойству 5) м. о.
\begin{eqnarray*}
cov(\xi, \eta) = E(\xi - E\xi) · E(\eta - E\eta) = 0.
\end{eqnarray*}
С учетом введенного обозначения для ковариации находим, что
\pause
\begin{eqnarray}\label{sum_of_dispers_0}
\boxed{
    D(\xi + \eta) = D(\xi) + D (\eta) + 2 cov (\xi , \eta).
    }
\end{eqnarray}
Если $\xi$ и $\eta$ независимы, то дисперсия суммы $\xi+\eta$ равна сумме дисперсий:
\begin{eqnarray}\label{sum_of_dispers}
\boxed{
D(\xi+\eta)=D\xi+ D\eta.
}
\end{eqnarray}

\end{frame}

\begin{frame}
\frametitle{Дисперсия}
\framesubtitle{Определение, свойства, ...}
Cвойство (\ref{sum_of_dispers}) остается выполненным и при меньшем предположении, нежели независимость $\xi$ и $\eta$: достаточно, чтобы величины $\xi$ и $\eta$ были некоррелированы, т. е. $cov(\xi , \eta) = 0$.
\end{frame}

\begin{frame}
\frametitle{Дисперсия}
\framesubtitle{Определение, свойства, ...}
\textbf{}{Замечание.}
Из некоррелированности $\xi$ и  $\eta$ вообще говоря, не следует их независимость.
\\
\\
\textbf{Пример}:  пусть случайная величина $\alpha$ принимает значения $0$, $\frac{\pi}{2}$ и $\pi$ с вероятностями $\frac{1}{3}$. Тогда $\xi=\sin \alpha$ и $\eta =\cos \alpha$
некоррелированы; в то же время они не только не независимы относительно вероятности $P$:
\begin{eqnarray*}
P(\eta =1, \xi=1) = 0 \neq  1/9 = P(\eta =1) P(\xi=1)
\end{eqnarray*}
Более того, $\xi$ и  $\eta$ - функционально зависимы: 
\begin{eqnarray*}
\eta ^ 2 + \xi ^ 2 = 1.
\end{eqnarray*}
\end{frame}

\begin{frame}
\frametitle{Дисперсия}
\framesubtitle{Определение, свойства, ...}
\textbf{}{Замечание.}
Из некоррелированности $\xi$ и  $\eta$ вообще говоря, не следует их независимость.
\\
\\
\textbf{Пример}:  пусть случайная величина $\alpha$ принимает значения $0$, $\frac{\pi}{2}$ и $\pi$ с вероятностями $\frac{1}{3}$. Тогда $\xi=\sin \alpha$ и $\eta =\cos \alpha$
некоррелированы; в то же время они не только не независимы относительно вероятности $P$:
\begin{eqnarray*}
P(\eta =1, \xi=1) = 0 \neq  1/9 = P(\eta =1) P(\xi=1)
\end{eqnarray*}
Более того, $\xi$ и  $\eta$ - функционально зависимы: 
\begin{eqnarray*}
\eta ^ 2 + \xi ^ 2 = 1.
\end{eqnarray*}
\end{frame}

\begin{frame}
\frametitle{Дисперсия}
\framesubtitle{Определение, свойства, ...}
Свойства (\ref{sum_of_dispers_0}) и (\ref{sum_of_dispers}) для $\xi_1, \cdots, \xi_n$:
\begin{eqnarray}
\boxed{
    D(\xi_1 + \dots + \xi_n) = D(\xi_1) + \dots + D (\xi_n) + 2 \sum_{i>j} cov (\xi_i , \xi_j).
    }
\end{eqnarray}
\pause
Если $\xi_1, \dots, \xi_n$ попарно независимы (попарно некоррелированы):
\begin{eqnarray}\label{sum_of_dispers}
\boxed{
D \left(\sum_{i=1}^n  \xi_i \right)= \sum_{i=1}^n D\xi_i.
}
\end{eqnarray}
\end{frame}

\begin{frame}
\frametitle{Дисперсия}
\framesubtitle{Определение, свойства, ...}
Свойства (\ref{sum_of_dispers_0}) и (\ref{sum_of_dispers}) для $\xi_1, \cdots, \xi_n$:
\begin{eqnarray}
\boxed{
    D(\xi_1 + \dots + \xi_n) = D(\xi_1) + \dots + D (\xi_n) + 2 \sum_{i>j} cov (\xi_i , \xi_j).
    }
\end{eqnarray}
\pause
Если $\xi_1, \dots, \xi_n$ попарно независимы (попарно некоррелированы):
\begin{eqnarray}\label{sum_of_dispers}
\boxed{
D \left(\sum_{i=1}^n  \xi_i \right)= \sum_{i=1}^n D\xi_i.
}
\end{eqnarray}
\end{frame}


