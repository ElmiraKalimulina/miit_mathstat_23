\begin{frame}
\frametitle{Примеры}
\framesubtitle{}

\begin{itemize}
	\item	<1-> $\xi$ - бернуллиевская сл.в., принимающая два значения $1$ и $0$ с вероятностями $p$ и $q$:
	<2->
	\begin{equation*}
	   D\xi= E(\xi - E\xi)^2=E(\xi - p)^2=(1-p)^2p+p^2q=pq.
	\end{equation*}
	\item   <3-> $\Rightarrow$ если $\xi_1, ..., \xi_n$ - н.о.р берн. сл.в., $S_n=\xi_1 +...+\xi_n$, то 
$DS_n=npq$.
\end{itemize}

\end{frame}

\begin{frame}
\frametitle{Пример}
\framesubtitle{Оптимальное линейное оценивание}

\begin{itemize}
	\item $\xi$ и $\eta$, наблюдается лишь сл.в. $\xi$, предполагается, что $\xi$ и $\eta$ - коррелированы. \pause $f=f(\xi)$ от $\xi$ называется \textit{оценкой} для $\eta$. \pause $f^*=f^*(\xi)$ от $\xi$  -  \textit{оптимальная в среднеквадратическом смысле оценка} для $\eta$, если:
		\begin{equation*}
	   E(\eta - f^*(\xi))^2= \inf_f E(\eta - f(\xi))^2.
	\end{equation*}
	\pause
	Найдём оптимальную оценку в классе линейных оценок
$\lambda(\xi) = a + b \xi$: $\Rightarrow (*)$
\end{itemize}

\end{frame}


\begin{frame}
\frametitle{Пример}
\framesubtitle{Оптимальное линейное оценивание}
$(*)$
\pause $g(a,b)=M(\eta - (a + b\xi))^2,$
\begin{eqnarray*}
\pause \frac{\partial g(a,b)}{\partial a}=-2M(\eta - (a + b\xi)),
\\
\pause \frac{\partial g(a,b)}{\partial b}=-2M \left[ (\eta - (a + b\xi))\xi \right],
\end{eqnarray*}
$\Rightarrow$ \pause
   \boxed{
   $$\lambda^*(\xi) = E \eta + \frac{cov (\xi, \eta)}{D \eta} (\xi - E\xi)$$%.
}
\pause
\\
Cреднеквадратическая ошибка оценивания равна:
   \boxed{
   $$\Delta^*=E( \eta -\lambda^*(\xi))^2 = D \eta( 1 - \rho^2 (\xi , \eta)$$%.
}
\end{frame}
